\documentclass[11pt,a4paper]{article}

\usepackage[utf8x]{inputenc}
\usepackage[T1]{fontenc}

\usepackage{pdfpages}
\usepackage{palatino}
\usepackage{enumitem}
\usepackage{outlines} %support nested list
\usepackage{float}
\usepackage{framed}

\usepackage{desclist}

\usepackage{todonotes}
%Define the listing package
\usepackage{listings} %code highlighter
\usepackage{color} %use color
\definecolor{mygreen}{rgb}{0,0.6,0}
\definecolor{mygray}{rgb}{0.5,0.5,0.5}
\definecolor{mymauve}{rgb}{0.58,0,0.82}
 
%Customize a bit the look
\lstset{ %
 backgroundcolor=\color{white}, % choose the background color; you must add \usepackage{color} or \usepackage{xcolor}
 basicstyle=\footnotesize, % the size of the fonts that are used for the code
 breakatwhitespace=false, % sets if automatic breaks should only happen at whitespace
 breaklines=true, % sets automatic line breaking
 captionpos=b, % sets the caption-position to bottom
 commentstyle=\color{mygreen}, % comment style
 deletekeywords={...}, % if you want to delete keywords from the given language
 escapeinside={\%*}{*)}, % if you want to add LaTeX within your code
 extendedchars=true, % lets you use non-ASCII characters; for 8-bits encodings only, does not work with UTF-8
 frame=single, % adds a frame around the code
 keepspaces=true, % keeps spaces in text, useful for keeping indentation of code (possibly needs columns=flexible)
 keywordstyle=\color{blue}, % keyword style
% language=Octave, % the language of the code
 morekeywords={*,...}, % if you want to add more keywords to the set
 numbers=left, % where to put the line-numbers; possible values are (none, left, right)
 numbersep=5pt, % how far the line-numbers are from the code
 numberstyle=\tiny\color{mygray}, % the style that is used for the line-numbers
 rulecolor=\color{black}, % if not set, the frame-color may be changed on line-breaks within not-black text (e.g. comments (green here))
 showspaces=false, % show spaces everywhere adding particular underscores; it overrides 'showstringspaces'
 showstringspaces=false, % underline spaces within strings only
 showtabs=false, % show tabs within strings adding particular underscores
 stepnumber=1, % the step between two line-numbers. If it's 1, each line will be numbered
 stringstyle=\color{mymauve}, % string literal style
 tabsize=2, % sets default tabsize to 2 spaces
 title=\lstname % show the filename of files included with \lstinputlisting; also try caption instead of title
}
%END of listing package%

\setlist[itemize]{topsep=3pt,after=\vspace{.5\baselineskip}}
\usepackage[left=3cm,right=3cm,top=3cm,bottom=3cm]{geometry}
\setlength{\parskip}{2mm}


\def\blurb{\textsc{Université catholique de Louvain\\
  École polytechnique de Louvain\\
  Pôle d'ingénierie informatique}}
\def\clap#1{\hbox to 0pt{\hss #1\hss}}%
\def\ligne#1{%
  \hbox to \hsize{%
    \vbox{\centering #1}}}%
\def\haut#1#2#3{%
  \hbox to \hsize{%
    \rlap{\vtop{\raggedright #1}}%
    \hss
    \clap{\vbox{\vfill\centering #2\vfill}}%
    \hss
    \llap{\vtop{\raggedleft #3}}}}%
\begin{document}

\begin{titlepage}
\thispagestyle{empty}\vbox to 1\vsize{%
  \vss
  \vbox to 1\vsize{%
    \haut{\raisebox{-5mm}{\includegraphics[width=2.5cm]{img/logo_ucl.pdf}}}{\blurb}{\raisebox{-5mm}{\includegraphics[scale=0.20]{img/ingi_logo.png}}}
    \vfill
    \ligne{\Huge \textbf{\textsc{LINGI2251}}}
     \vspace{5mm}
    \ligne{\huge \textbf{\textsc{Software Engineering: Development Methods}}}
     \vspace{15mm}
    \ligne{\Large \textbf{\textsc{Assignment 2}}}
    \vspace{5mm}
    \ligne{\large{\textsc{12 april 2016}}}
    \vfill
    \vspace{5mm}
    \ligne{%
         \textsc{Alexandre Hauet\\Tanguy Vaessen} 
      }
      \vspace{5mm}
    }%
  \vss
  }
\end{titlepage}

%\tableofcontents

\section{Architectural Design}
\subsection{Hierarchical decomposition}

\begin{figure}[H]
 \centering
 \includegraphics[width=\textwidth]{../DecompositionView.png} 
 \caption{Hierarchical decomposition}
 \label{fig:dep}
\end{figure}

\subsection{Roles and interactions of all components}

Our system is composed of three parts. The first is the station controller which controls all the station. The second is the Gas Pump which is about the logistic of the pump. And finally, the Management part which allows to customers to pay or to manage the billing.\\


We use also  a model-view-controller in our design. We distinct the application into three parts :
\begin{enumerate}
\item{Model :} Manipulation of data
\item{View :} Interaction with the users
\item{Controller :} Management of the different events in the program
\end{enumerate}

\subsubsection*{Station controller}
The Station controller represents the global controller of all the gaz station.


\subsubsection*{Gas Pump}
The Gas pump represents the controller of a gas pump.
\begin{itemize}
\item{Gaz pump UI :} Interface with which the user interacts.
\item{Gaz pump controller :} Controls the different interaction between the system (Station controller) and the different elements of the Gas pumps.
\item{Fuel stock controller :} Manages the stock of fuel of the station.
\item{Fuel :} Represents a specific type of fuel.
\item{Gaz pump :} Delivers the gaz to the costumers and compute the accounting. 
\end{itemize}
\subsubsection*{Management}
The Management represents the management part of the application in which the payment for example while be manage.
\begin{itemize}
\item{Cashier UI :} Interface with which the cashier interacts.
\item{Cashier controller :} Allows the interaction between the cashier system and the global system (Station controller).
\item{Credit card controller :} Manages the interaction between the external credit card system and our system.
\item{Billing storage controller :} Manages the billing accounts.
\item{Payment controller :} Allows the customer to make a payment.
\item{Customer :} Represents a customer with the different fields needed
\item{Account :} Computes the account of a purchase
\item{Credit transaction :} Represents a transaction with a credit card. 
\item{Credit payment :} Allows to pay with the credit card
\item{Credit purchase :} Adds the purchase to the credit account.
\item{Account purchase :} Payments of the balance of the account 
\item{Cash purchase :} Managements of the purchase with cash
\end{itemize}

\section{Detailed Design}
\subsection{Our design}
\subsubsection{Class diagram}

\begin{figure}[H]
 \centering
 \includegraphics[width=\textwidth]{../ClassDiagram.png}
\end{figure}

\subsubsection{Description of classes}
\todo[inline]{Tanguy TODO}


\subsubsection{Uses diagram}

\begin{figure}[H]
 \centering
 \includegraphics[width=\textwidth]{../useDiagram.png}
\end{figure}
There are 2 loops in this Uses diagram:
\begin{itemize}
	\item between \textit{Gas pump UI} and \textit{Gas pump controller}
	\item between \textit{Cashier UI} and \textit{Cashier controller}
\end{itemize}
But this is not really a problem because there are just between a view and its controller. It makes senses that the controller uses the view to update it and that the view uses the controller to notify it when an user's action occurs. We think that they can no be eliminated.


\subsection{Design Patterns}
The skeletons are written in Java.

\subsubsection{Template Method}
\begin{figure}[H]
 \centering
 \includegraphics[width=0.75\textwidth]{../templateMethod.png}
\end{figure}

\lstinputlisting[language=java, inputencoding=utf8]{../templateMethod.java}

\subsubsection{Strategy Pattern}
\begin{figure}[H]
 \centering
 \includegraphics[width=0.8\textwidth]{../strategyPattern.png}
\end{figure}

\lstinputlisting[language=java, inputencoding=utf8]{../strategyPattern.java}


\subsubsection{Decorator Pattern}
\begin{figure}[H]
 \centering
 \includegraphics[width=0.35\textwidth]{../decoratorPattern.png}
\end{figure}

\lstinputlisting[language=java, inputencoding=utf8]{../decoratorPattern.java}


\subsubsection{Observer Pattern}
\begin{figure}[H]
 \centering
 \includegraphics[width=0.9\textwidth]{../observatorPattern.png}
\end{figure}

\lstinputlisting[language=java, inputencoding=utf8]{../observatorPattern.java}


\subsubsection{Composite Pattern}
\begin{figure}[H]
 \centering
 \includegraphics[width=0.8\textwidth]{../compositePattern.png}
\end{figure}

\lstinputlisting[language=java, inputencoding=utf8]{../compositePattern.java}

We can see in the snippet of code above where the new methods are implemented.
But the existing methods (the getters and setters for the bank account in this case) are implemented in the SingleAccount class. If we had other pre-existing methods in the Account, they would be implemented in the new SingleAccount class and also in the CompanyAccount. To call those methods, the developer should have a reference to the concrete object, not the new Account interface (because it does not define other methods than \textit{getRemainingBalance()}.

\end{document}
